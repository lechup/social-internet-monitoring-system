% trochę domyślnego formatowania dokumentu
\documentclass[a4paper]{article}
\usepackage{fullpage}

\usepackage[utf8]{inputenc}
\usepackage{polski}
\usepackage[polish]{babel}
\usepackage{userstory}

\begin{document}
    \begin{userstory}{Główne zadanie systemu}
        Tutaj opisać ogólnie główne zadanie systemu, bez realizacji którego nie będzie można powiedzieć, że system działa. Jak system składa się z kilku podsystemów, dla każdego systemu przygotować podobną kartę wymagań.
%        Użytkownik za pomocą smartfona wyposażonego w system Android oraz kamerę lub za pomocą przeglądarki na komputerze stacjonarnym i kamery podłączonej do niego,\\*
%        ma możliwość udostępnienia on-line aktualnego obrazu i dźwięku
        \begin{questions}
            \item{
                Jakie jest główne zadanie systemu?
            }
            \item{
                Kto jest użytkownikiem systemu?
            }
            \item{
                Czy jest wymóg użycia konkretnej technologii?
            }
            \item{
                Jakie urządzenia muszą współpracować z systemem?
            }
            \item{
                Jaka jest wymagana skalowalność systemu?
            }
            \item{
                Czy są jakieś wymagania dotyczące środowiska produkcyjnego w jakim ma działać system?
            }
            \item{
                Czy są jakieś specjalne wymagania, które nie wynikają z funkcji jakie powinien posiadać system?
            }
        \end{questions}
    \end{userstory}
\end{document}
