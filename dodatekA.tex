\chapter{\LaTeX, środowisko \texttt{userstory}}
\label{cha:dodatekA}

Na potrzeby niniejszej pracy autor musiał wymyślić sposób na cyfryzację kart wymagań, związanych z nimi testów akceptacyjnych oraz ekranów, tak aby w łatwy sposób dało się je dołączyć do pracy. Tak powstał tzw. ,,package'' udostępniający środowisko \texttt{userstory}. Więcej o dodatkowych zaletach cyfryzacji można przeczytać \namedref{sec:ZSWcyfryzacja}.

Należy pamiętać, że każda metodyka zwinna jest przeciwna generowaniu bezużytecznej dokumentacji. Należy więc wziąć to pod uwagę przy wykorzystywaniu środowiska ,,userstory''.

\section{Sposób użycia}
\label{sec:dodatekAsu}

Aby skorzystać ze środowiska należy wykonać dwie rzeczy:
\begin{itemize}
    \item do katalogu dokumentu \LaTeX wgrać plik \texttt{userstory.sty}
    \item do generowanego dokumentu dołączyć nagłówek:
    \begin{lstlisting}
    \usepackage{userstory}
    \end{lstlisting}
\end{itemize}

Po takim zabiegu w kodzie dostępne jest środowisko \texttt{userstory}. W jego wnętrzu zapisujemy treść karty wymagań, dodatkowo mamy do dyspozycji:
\begin{itemize}
    \item Środowisko \texttt{tests} do definiowania nowych testów akceptacyjnych. Za pomocą komendy \texttt{\\item} możemy zdefiniować w nim kolejny test akceptacyjny.
    \item Środowisko \texttt{questions} do definiowania pytań do klienta, które pojawiły się w ,,międzyczasie''. Za pomocą komendy \texttt{\\item} możemy zdefiniować w nim kolejne pytanie.
    \item Dwuargumentową komendę \texttt{\\src{\$file}{\$caption}} do umieszczania ekranów w dowolnym miejscu środowiska (przy pytaniach, przy samej karcie wymagań albo przy konkretnym teście akceptacyjnym). Argument \textbf{\$file} to ścieżka do pliku graficznego, a \textbf{\$caption} to opis który będzie dodany pod ekranem.
\end{itemize}

\section{Przykład użycia}
\label{sec:dodatekApu}

Poniżej przygotowany przykład wykorzystania środowiska \texttt{userstory} celem przygotowania karty wymagań ,,Logowania użytkownika''.
\lstinputlisting{userstory-latex/logowanie-uzytkownika.tex}

\section{Kod środowiska}
\label{sec:dodatekAks}

Poniżej znajduje się kod wykorzystywany w celu definicji środowiska.
\lstinputlisting{userstory-latex/userstory.sty}

\section{Szablon ,,Główne zadanie systemu''}
\label{sec:dodatekAsgzs}

Poniżej znajduje się kod odpowiedzialny za wygenerowanie szablonu głównego zadania systemu.
\lstinputlisting{userstory-latex/szablon-glowne-zadanie-systemu.tex}
