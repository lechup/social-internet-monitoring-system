\newpage
%---------------------------------------------------------------------------
\chapter[Wprowadzenie do zwinnej metodyki rozwoju oprogramowania \textit{Extreme Programming} (XP)]{Wprowadzenie do zwinnej metodyki rozwoju oprogramowania \textit{Extreme Programming} (XP)\footnote{Głównym źródłem informacji na którym  ten rozdział jest książka \cite{Bec99}. Autor tej książki uważany jest za twórcę metodyki \textit{XP}.}}
\label{cha:ZMTO}

\begin{center}
    ,,Wytwarzając oprogramowanie i pomagając innym w tym zakresie,\\*
    odkrywamy lepsze sposoby wykonywania tej pracy.\\*
    W wyniku tych doświadczeń przedkładamy:\newline
    \newline
    \textbf{Ludzi i interakcje} ponad procesy i narzędzia.\\*
    \textbf{Działające oprogramowanie} ponad obszerną dokumentację.\\*
    \textbf{Współpracę z klientem} ponad formalne ustalenia.\\*
    \textbf{Reagowanie na zmiany} ponad podążanie za planem.\newline
    \newline
    Doceniamy to, co wymieniono po prawej stronie,\\*
    jednak bardziej cenimy to, co po lewej.''
\end{center}
\hfill \begin{small}\textit{--- Kent Beck et al., http://agilemanifesto.org}\end{small}

Przy tworzeniu projektu wchodzącego w skład niniejszej poczyniono kilka założeń. Większość z nich dotyczy sposobu prowadzenia projektu -- dokumentacji, testów, komunikacji w grupie. W następnym punkcie znajduje się lista tych założeń.

Projekt nie spełniający ich, nie będzie mógł być prowadzony zwinnie. Listę można traktować jako swojego rodzaju zbiór dobrych praktyk metodyki rozwoju oprogramowania zwanej \textit{Extreme Programming} lub w skrócie \textit{XP} -- Programowanie Ekstremalne. Oczywiście ze względu na samodzielny charakter pracy, autor nie mógł sprawdzić wszystkich założeń w praktyce.

\section{Założenia}
\label{sec:ZMTOzalozenia}

Niżej przedstawiono podstawowe założenia metodyki \textit{Extreme Programming} bez których metodyka nie tyle nie będzie działała, co nie będzie można powiedzieć, że w projekcie zastosowana jest metodyka \textit{XP}.

\begin{description}
    \item[Klient jest zawsze dostępny] Osoba która wie jak system ma działać od strony użytkownika jest zawsze dostępna dla programistów, forma komunikacji nie jest narzucona, natomiast preferowana jest twarzą w twarz. Osoba ta nie jest potrzebna tylko na początku projektu lecz \emph{przez cały okres} jego tworzenia i rozwoju.
    \item[Klient ma zawsze możliwość zmiany] Nie ma rzeczy której nie da się zmienić w systemie, klient może w każdym momencie zmienić wymagania systemu. Grupa projektowa musi reagować na te zmiany.
    \item[Projekt jest prowadzony w krótkich iteracjach] Jedno do trzy-tygodniowe okresy czasu, po których klient otrzymuje chciane funkcjonalności, wykonania których podjęliśmy się w zadanym okresie czasu.
    \item[Testowanie automatyczne] Każdy nowy element systemu musi posiadać napisane testy automatyczne (jednostkowe, funkcjonalne) -- system można w każdym momencie automatycznie przetestować. Najlepiej testy pisać przed implementacją nowego elementu.
    \item[Projektujemy, planujemy zawsze proste minimum] Nie powinno planować się na wyrost. Należy określać minimum, które będzie spełniało wymagania klienta w danej iteracji. Zawsze powinno trzymać się jak najmniejszą złożoność systemu, nawet kosztem dodatkowych zmian.
    \item[Ciągła integracja] Poprawki nanoszone są cały czas na istniejący, system. Integracja następuje wręcz kilka razy dziennie. Wymagany jest reżim testowy -- integrowany jest tylko ten kod, który ma napisane testy automatyczne oraz nie blokuje wykonania żadnego z dotychczasowo napisanych testów.
    \item[Brak specjalizacji] Żadna osoba z grupy projektowej nie powinna specjalizować się w jakiejś funkcji (programista, architekt, integrator, analityk). Wszyscy biorą czynny udział we wszystkich aspektach projektu.
    \item[Grupa prowadząca projekt nie jest zbyt liczna] Grupa w której tworzony jest projekt lub pod-projekt nie może być zbyt liczna, ułatwia to komunikację. Każdy większy projekt da się podzielić na szereg mniejszych pod-projektów. Nie ma określonego limitu górnego, jeżeli występują problemy w komunikacji najczęściej grupa projektowa jest zbyt liczna.
    \item[Komunikacja twarzą w twarz] Preferowaną formą komunikacji w grupie projektowej jest komunikacja twarzą w twarz. Tylko przy takiej rozmowie uczestnicy projektu nie są rozpraszani przez inne rzeczy i mogą skupić się na zadanym temacie. Taki sposób komunikacji ułatwia wynajdywanie błędów i niejasności we wczesnej fazie projektu.
    \item[Wspólne programowanie, projektowanie] Faworyzuje się tworzenie wszystkich elementów systemu w parach. Zapewniając w ten sposób ciągłe badanie jakości kodu, czy też poprawności architektury systemu.
    \item[Architektura jest zmienna] Architektura jest czymś co się zmienia wraz z rozwojem projektu i funkcjonalności wymaganej przez klienta. Może się zmienić na każdym etapie projektu.
    \item[Kod i testy są dokumentacją] Nie ma prowadzonej dodatkowej dokumentacji implementacyjnej np. dla programistów. Dobrze napisane testy, kod z komentarzami oraz ostatecznie rozmowa z innymi uczestnikami projektu jest najlepszą dokumentacją aktualnego stanu systemu.
\end{description}

\section{Zalety}
\label{sec:ZMTOzalety}

Metodyka \textit{Extreme Programming} zmienia całkowicie podejście do sposobu tworzenia oprogramowania. Głównie przez umieszczenie klienta, w centrum prowadzonego projektu i jego ciągłe zaangażowanie na całym etapie tworzenia i rozwoju systemu. To on może w dowolnym momencie wprowadzić praktycznie dowolne zmiany. Zyski jakie daje metodyka zwinna \textit{XP} można rozpatrywać na wielu płaszczyznach:

\begin{description}
    \item[Koszty]{Metodyka daje nowe możliwości związane z liczeniem kosztów (per iteracja, per funkcjonalność). Przez krótkie iteracje i działające oprogramowanie łatwiej rozliczać się z klientem, a sam klient chętniej płaci. Długoterminowo unikamy kosztów wynikających z błędów początkowego projektowania, czy też zmian w projekcie, których nie jesteśmy w stanie uniknąć.}
    \item[Czas]{Zarządzanie czasem w krótkich terminach iteracyjnych jest dużo bardziej wydajne i bardziej przewidywalne niż planowanie całego projektu od początku do końca. Brak konkretnej daty zakończenia projektu umożliwia uniknięcie ,,marszów śmierci'', czy przesuwania terminów. Klient sam zarządza tym co ma być zrobione i jakim kosztem czasowym jest to obarczone. Reżim testowy i ciągła integracja powoduje, że oprogramowanie dostarczane jest szybko, i co bardzo ważne dla klienta -- jest to oprogramowanie spełniające jego aktualne wymagania.}
    \item[Jakość]{Ciągła integracja i obecność testów powoduje, że pomimo możliwych ciągłych zmian kod w większości przypadków działa, czyli jest wysokiej jakości. Jakość kodu gwarantowana jest również przez programowanie w parach. Zastępuje ono cykliczne sprawdzanie kodu oraz umożliwia uniknięcie problemów związanych z błędną architekturą systemu, które najczęściej pojawiają się bardzo późno.}
    \item[Zakres]{Bardzo łatwo zmienić zakres systemu w trakcie jego tworzenia, wymaga się jedynie ustalenia w której iteracji ma być on zwiększony lub pomniejszony. Jako, że nie mamy ustalonych terminów całości projektu, zmiana zakresu nie jest dla nas.}
    \item[Brak specjalizacji]{Ciągła komunikacja werbalna w projekcie zapobiega specjalizacji poszczególnych członków zespołu. Każdy powinien wiedzieć jak najwięcej o systemie. Takie podejście gwarantuje niezależność w przypadku odejścia członka ,,specjalisty''.}
    \item[Porażka projektu]{Szybkie dostarczanie działającego, funkcjonalnego oprogramowania ma duży wpływ na zadowolenie klienta. Przekłada się to bezpośrednio na poczucie satysfakcji oraz pozytywną atmosferę wśród członków zespołu. Dzięki temu ludzie są mniej skłonni do zmiany miejsca zatrudnienia ze względu na stres, ciągłe przekraczanie terminów czy brak satysfakcji z wykonywanej pracy.}
\end{description}

\section{Wady}
\label{sec:ZMTOwady}

Jak każda metodyka, także i ta zwinna posiada swoje wady. Warto znać je przed rozpoczęciem projektu, w którym planuje się ją wykorzystać. Główne z nich są wymienione niżej.

\begin{description}
    \item[Ciągła dostępność klienta]{Klient lub osoba wyznaczona przez niego, która zna potrzeby systemu musi być dostępna programistom ,,na zawołanie'' i ściśle z nimi współpracować.}
    \item[Konsekwencja]{Decydując się na \textit{XP} nie możemy zrezygnować z założeń (wspomniane szczegółowo \namedref{sec:ZMTOzalozenia}). Brak konsekwencji spowoduje w większości przypadków porażkę projektu.}
    \item[Brak konkretnej daty zakończenia projektu]{Podejście iteracyjno-wymaganiowe powoduje, że znany jest jedynie czas zakończenia aktualnej iteracji w skład której wchodzą konkretne karty wymagań. Ciągła możliwość zmiany zakresu i funkcjonalności systemu powoduje, że nie można określić daty zakończenia projektu.}
\end{description}

\section{Kiedy nie stosować metodyki zwinnej}
\label{sec:ZMTOknsm}

Czasami metodyki \textit{Extreme Programming} nie można zastosować ze względu na charakter prowadzonego projektu lub przyzwyczajenia klienta. Główne wskazówki kiedy \textit{XP} \textbf{nie nadaje się} jako metodyka prowadzenia projektu opisane są poniżej.

\begin{description}
    \item[Brak dostępności klienta]Jeżeli wiadomo, że klient nie będzie brał czynnego udziału w projekcie. Doprowadzi to do sytuacji, w której tworzone oprogramowanie będzie oprogramowaniem niezgodnym z wymaganiami klienta.
    \item[Duża grupa projektowa] Grupa projektowa \textit{XP} nie powinna być większa niż 10 osób. Więcej osób generuje problemy w komunikacji. Częściowym rozwiązaniem może być podział systemu na podsystemy, przygotowywane przez kilka mniejszych grup projektowych.
    \item[Długie wykonywanie testów] Kompilacja i wykonanie testów automatycznych trwa dłużej niż dzień (24 godziny). Co praktycznie uniemożliwia systematyczną ciągłą integrację.
    \item[Brak środowiska testowego innego niż produkcyjny] W sytuacji kiedy koszty środowiska testowego są bardzo wysokie, klient nie będzie chciał wydawać drugi raz dużych pieniędzy. Może się też zdarzyć, że z innych względów nie będziemy w stanie zagwarantować środowiska testowego, co automatycznie wyklucza testy automatyczne i ciągłą integrację.
\end{description}

\section{Cykl życia idealnego projektu}
\label{sec:ZMTOcykl}

Każda metodyka ma etapy w jakim może znaleźć się projekt. W tym rozdziale przybliżono idealny cykl życia projektu \textit{XP}.

\begin{packed_enum}
    \item \textbf{Badanie (Exploration).} Faza w której pobiera się od klienta główne wymagania systemu, testuje możliwe do wykorzystania technologie, bada się ich wydajność i użyteczność w projekcie. Jeżeli wykorzystywana technologia jest nowa dla zespołu, testuje się jej wykorzystanie na jakimś małym projekcie powiązanym z docelowym systemem.
    \item \textbf{Planowanie (Planning).} Faza w której grupa projektowa szacuje z klientem terminy głównych wymagań systemu. Klient określa priorytet, a grupa projektowa czas potrzebny na wykonanie poszczególnych kart wymagań.
    \item \textbf{Iteracje do pierwszego wydania (Iterations to First Release).} Faza w której w iteracjach jedno do trzy-tygodniowych przygotowuje się system zgodnie z kartami wymagań. Koniec każdej iteracji powinien być ,,małym świętem'' -- darmowa pizza, wolny dzień w pracy etc.
    \item \textbf{Wdrażanie do produkcji (Productionizing).} Faza w której system jest wdrażany do produkcji.Na tym etapie należy pamiętać o sposobie na łatwą integrację zmian -- zarówno kodu jak i migracji danych oraz systemie testowym. Pod koniec tego etapu należy zrobić huczną imprezę!
    \item \textbf{Utrzymanie i konserwacja (Maintenance).} Jest to najdłuższa faza projektu \textit{XP}. Można by właściwie powiedzieć jego naturalny stan. W tej fazie iteracyjne, wdrażane są szczegółowe wymagania klienta, czy też nowe funkcjonalności odpowiadające jego aktualnym potrzebom.
    \item \textbf{Zakończenie (Death).} Jest to faza w której projekt uznawany jest za zakończony. Może to być sytuacja idealna, w której klient nie ma potrzeby dalszej zmiany oprogramowania -- dostarczono mu produkt idealny. Może to być również mniej przyjemna sytuacja -- system nie jest w stanie spełnić nowych wymagań stawianych przez klienta. Tak czy inaczej, dobrym pomysłem jest zorganizowanie ,,stypy'' na której trzeba poruszyć problem, tego: ,,Co można było zrobić lepiej?'' i wyciągnąć na tej podstawie wnioski mogące się przydać w nowym projekcie.
\end{packed_enum}

\section{Zwinna specyfikacja wymagań}
\label{cha:ZMTOzwinnaSpecyfikacjaWymagan}

Do tej pory nie znaleziono idealnego, uniwersalnego rozwiązania na pobieranie wymagań od klienta. W większości przypadków trzeba pewnego czasu, aby wypracować wspólny język z klientem i nauczyć się razem z nim współpracować przy akwizycji wymagań.

Podchodząc zwinnie do projektowania systemu na początku należy pobrać od klienta tylko te wymagania, które są konieczne do osiągnięcia minimalnych celów biznesowych -- głównego zadania systemu. Wszystkie sprawy poboczne, należy oczywiście zapisywać, lecz zostawić na później, aż projekt przejdzie do etapu \textit{Utrzymanie i konserwacja} (więcej \namedref{sec:ZMTOcykl}).

Celem zwinnej specyfikacji wymagań jest stworzenie wymagań \textbf{minimalnego systemu}, spełniającego \textbf{wszystkie podstawowe funkcje biznesowe} zdefiniowane przez klienta, który można \textbf{wdrożyć, używać i rozwijać}.

\subsection{Narzędzia}
\label{sec:ZSWnarzedzia}

Zwinna specyfikacja wymagań \textit{Extreme Programming} posiada szereg prostych narzędzi, które upraszczają proces jej przygotowania. Są one wraz z opisami wymienione poniżej.

\begin{description}
    \item[User Stories]{W \cite{Mad09} określane jako karty wymagań. Są to krótkie, ogólnikowe historyjki opisujące jakąś funkcjonalność systemu.}
    \item[Screens]{W \cite{Mad09} określane jako ekrany. Są to szkice obrazujące ogólny układ przycisków, formularzy, tabel mogących pojawić się w karcie wymagań}
    \item[Acceptance Tests]{W \cite{Mad09} określane jako testy akceptacyjne/scenariusze. Są to opisy słowne prawidłowego zachowania systemu dotyczące kart wymagań, swojego rodzaju scenariusz karty wymagań na bazie których programiści mogą stworzyć testy (o testach akceptacyjnych w \textit{XP} można więcej przeczytać w \cite{Jef00} pod hasłem \textit{Acceptance tests}}
\end{description}

Dokumenty te nie mają konkretnej formy. Zwinna metodyka pozwala dostosować ją do własnych wymagań, które będą działały najlepiej dla wykorzystujących metodykę programistów oraz ich klienta. Karty wymagań, scenariusze oraz ekrany należy spisywać na małych karteczkach jednego formatu. Przygotowując powyższe ,,dokumenty'' należy pamiętać o poniższych zasadach.

\begin{description}
    \item[Prostota]Złożone scenariusze czy karty wymagań należy rozbić na kilka mniejszych, dzielenie musi być zrobione z klientem w formie rozmowy -- nigdy przez samego programistę. Karty wymagań powinny być do wykonania w ciągu jednego do dwóch tygodni czasu programisty.
    \item[Klient jest autorem kart wymagań jak i testów akceptacyjnych/scenariuszy]Programista może jedynie pokazać jak należy pisać kartę wymagań, tak żeby wypracować z klientem wspólny język, może pomagać w podzieleniu złożonych wymagań na mniejsze. Dopuszcza się sugerowanie możliwych scenariuszy, natomiast sama reakcja systemu musi być już określona przez klienta
    \item[Ekrany powinny mieć swoją sekwencję]Jak istnieje sekwencja ekranów, należy pamiętać o nadaniu numerów odpowiadających kolejności ich występowania.
\end{description}

\subsection{Cyfryzacja i zarządzanie}
\label{sec:ZSWcyfryzacja}

W związku z potrzebą dołączenia kart wymagań, związanych z nimi testów akceptacyjnych oraz ekranów do niniejszej pracy, trzeba było wymyślić sposób na ich cyfryzację. Dodatkową zaletą formy elektronicznej jest fakt, że ułatwia ona wymianę informacji pomiędzy uczestnikami projektu, pomoże uniknąć sytuacji zagubienia lub zniszczenia jakiegoś elementu kart wymagań, ułatwi ich zarządzaniem i przetrzymywaniem np. w wykorzystywanym systemie kontroli wersji.

Wykorzystywanie \LaTeX~u do składu pracy nakierowało na pomysł stworzenia szablonu \LaTeX, który można by było wykorzystać celem konwersji kart wymagań i ich elementów analogowych do dokumentu cyfrowego. Łatwe oddzielenie warstwy prezentacji od samej treści oraz możliwość eksportu do formatu PDF umożliwiałoby również tworzenie w prosty sposób ładnego dokumentu dla klienta po spotkaniach na których pobierane są od niego wymagania.

Tak powstał tzw. \textit{package} udostępniający środowisko \texttt{userstory}. Szczegółowy opis tego środowiska znajduje się \namedref{cha:dodatekA}.

\section{Zwinna analiza wymagań}
\label{sec:ZMTOzwinnaAnalizaWymagan}

Zwinna analiza wymagań ma miejsce w fazie \textit{Badanie (Exploration)} projektu (więcej \namedref{sec:ZMTOcykl}), po tym jak ustalono z klientem główne zadanie systemu oraz sporządzono dla niego wszystkie karty wymagań. Cele tej fazy są wymienione poniżej.

\begin{description}
    \item[Testowanie technologii]{Poprzez budowanie prototypów, testuje się wykonalność oraz wydajność konkretnej technologii. Testy służą wybraniu najbardziej odpowiedniej technologii. Jeżeli uruchomienie jakiejś technologii trwa dłużej niż tydzień, należy zastanowić się nad alternatywami.}
    \item[Testowanie architektury]{Najczęściej system można zbudować na kilka sposobów, z pomocą może przyjść znowu prototypownie. W grupach 2-3 osobowych tworzone są równolegle 2-3 podejścia do architektury systemu. Porównanie umożliwia wybranie tej najlepszej.}
    \item[Poznanie limitów wydajnościowych rozwiązania]{Bardzo ważnym czynnikiem, który trzeba wziąć pod uwagę jest wydajność i odnieść ją wymagań stawianych przez klienta.}
    \item[Testowanie skalowalności]{Równie ważnym czynnikiem jest sposób w jaki skaluje się projektowane rozwiązanie. Zapewnienie stałej, liniowej, bądź logarytmicznej skalowalności jest ideałem.}
    \item[Wybór narzędzi umożliwiających ciągłą integrację]{Jednym z ważniejszych wymagań \textit{XP} jest zdolność do ciągłej integracji. Należy wybrać system kontroli wersji kodu źródłowego, narzędzie do przeprowadzania testów automatycznych oraz narzędzie umożliwiające łatwą i pewną migrację danych.}
    \item[Wybór narzędzi wspomagających prowadzenie projektu]{Powinno się również pomyśleć o narzędziach wspomagających zarządzanie projektem, czy też ułatwiającymi komunikację między uczestnikami projektu.}
\end{description}

Zgodnie z założeniami \textit{XP} (więcej \namedref{sec:ZMTOzalozenia}), wszystkie ustalenia sporządzone w tej fazie mogą ulec zmianie na każdym późniejszym etapie prowadzenia projektu. Zwinna analiza wymagań nie zamyka możliwości późniejszej zmiany technologii, architektury czy dowolnego z elementów wspomagających projekt.

\newpage
%---------------------------------------------------------------------------
