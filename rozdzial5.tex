\chapter{Wnioski i podsumowanie}
\label{cha:WnioskiIpodsumowanie}

W tym rozdziale umieszczono podsumowanie oraz wnioski wyciągnięte podczas pisania pracy. Podzielono je na dwa podpunkty, zgodnie z głównymi celami pracy.

\section{Zwinna metodyka \textit{Extreme Programming}}
\label{sec:WnioskiZwinnaMetodyka}

W dzisiejszym świecie bardzo ważną sprawą jest szybkie reagowanie na zmienne wymagania klienta i jego biznesu. Oprogramowanie jest więc tworem żywym, ciągle rozwijanym i usprawnianym. Metodyki zwinne, w tym także \textit{Extreme Programming} idealnie radzą sobie z projektami tworzonym w takim środowisku.

Świadomy i konsekwentny wybór metodyki zwinnej prowadzenia projektu jest opłacalny zarówno dla klienta (jakość, koszty, możliwość ciągłych zmian) jak i dla grupy prowadzącej projekt (miła atmosfera, brak hierarchii w zespole, zadowolenie). Więcej informacji na ten temat można znaleźć \namedref{sec:ZMTOzalety}.

Zaproponowane zwinne dokumenty specyfikacji (\namedrefw{cha:ZMTOzwinnaSpecyfikacjaWymagan}) oraz analizy (\namedrefw{sec:ZMTOzwinnaAnalizaWymagan}) wymagań sprawdziły się podczas projektowania ,,Społecznościowego, internetowego systemu monitoringu''.

Jedyną negatywną sprawą sprawą związaną bezpośrednio z cyfryzacją specyfikacji wymagań były szkice ekranów. Dużo lepszym rozwiązaniem niż przygotowywanie ich za pomocą edytora graficznego, wydaje się ręczny szkic na kartce papieru lub tablicy. Jeżeli istnieje potrzeba cyfryzacji takiego ekranu, z pomocą może przyjść telefon komórkowy z aparatem. Takie rozwiązanie wydaje się dużo mniej uciążliwe i przy tym daje dużo więcej satysfakcji.

\section{Implementacja prototypu}
\label{sec:WnioskiImplementacjaPrototypu}

Po uciążliwej i niełatwej, ale ostatecznie udanej implementacji prototypu systemu, można w pełni świadomie powiedzieć, że ,,Społecznościowy, internetowy system monitoringu'' jest systemem do zrealizowania z wykorzystaniem zasugerowanych technologii.

Przed publiczną odsłoną systemu wymagane jest jego usprawnienie -- o którym wspomniano \namedref{sec:ImplementacjaPrototypuCoWymagaUsprawnienia}. Szczególnie ważną informacją, z którą również należy się zapoznać przed korzystaniem z jakiegokolwiek elementu systemu, jest informacja dotycząca bezpieczeństwa (\namedref{sec:ImplementacjaPrototypuBezpieczenstwo}).

Największe nadzieję wiązane są z technologią P2P Multicasting wbudowaną w Adobe Flash Platform. Sprawdziła się praktycznie we wszystkich aspektach w związku z którymi została użyta. Autorzy rozwiązania \cite{MattKauf2009} twierdzą, że technologia jest skalowalna od tysięcy do nawet miliona użytkowników. Niestety ze względów na brak środków technicznych i finansowych, nie można było przeprowadzić testów na tak dużą skalę. Udało się natomiast wykonać pomniejszy test wydajności (\namedref{sec:ImplementacjaPrototypuWydajnoscP2P}) z którego wnioski umieszczono poniżej:

\begin{packed_item}
    \item{Im gorsze łącze Mobile Broadcastera, tym gorsza możliwa jakość przesyłu wideo.}
    \item{Im gorsze połączenie Mobile Broadcastera, a ilość Browser Reveicerów większa tym lepsza płynność transmisji wideo.}
    \item{Im lepsze połączenie Mobile Broadcastera tym lepsza możliwa jakość przesyłu wideo.}
    \item{Im więcej klientów Browser Reveivera podłączone do streamu tym większe możliwe opóźnienie dla pojedynczego Browser Receivera.}
\end{packed_item}

Oczywiście technologia Adobe posiada kilka wad. Pod koniec wykonywania projektu okazało się, żeby stworzyć aplikację działającą pod systemem iOS za pomocą Adobe AIR for Mobile, należy posiadać konto Apple Developer Account. Jest to konto, którego założenie jest płatne. Nie jest to bezpośrednio wina firmy Adobe, lecz polityki firmy Apple.

Kolejną ułomnością technologii Adobe jest fakt, iż komunikacja P2P odbywa się przy udziale wysokiego portu UDP. Komunikacja ta może być blokowana w sieciach korporacyjnych, co uniemożliwia jej zastosowanie dla 100\% użytkowników.

Powyższe wady nie zmieniają jednak faktu, że któregoś dnia technologia ta może spowodować stworzenie globalnego i praktycznie nieograniczenie skalowalnego ,,Społecznościowego, internetowego systemu monitoringu'' dostępnego dla każdego z nas.

\newpage
%---------------------------------------------------------------------------
