\chapter{Testy i omówienie wyników}
\label{cha:TestyIomowienieWynikow}
Zgodnie z filozofią metodyki \textit{XP} (szczegóły \namedref{sec:ZMTOzalozenia}), dokumentacja projektu sprowadza się do opisu architektury oraz komentarzy kodu, testów oraz komentarzy do testów. Unika się tworzenia bazy wiedzy, która najczęściej jest po pewnym czasie nieaktualizowana.

Dlatego też, aby zapoznać się z działaniem systemu należy przejrzeć architekturę systemu, a następnie zapoznać się bezpośrednio z kodem źródłowym i zawartymi w nim komentarzami.

\section{Uruchomienie systemu}

Aby uruchomić system z kodu źródłowego należy przygotować szereg aplikacji i skonfigurować odpowiednio środowisko. Niżej prezentowana jest lista części systemu, wraz z opisem jak daną część uruchomić.

\begin{description}
    \item[Interfejs WWW] Uruchomić aplikację Django, znajdującą się w dołączonym kodzie źródłowym w katalogu \texttt{./src/Django/facewithme/}. Wymagane do uruchomienia biblioteki środowiska Pythonowskiego w formacie pliku wymagań programu PIP znajdują się w \texttt{./src/Django/facewithme/requirements/requirements.txt}. Instrukcja instalacji i wdrożenia aplikacji Django znajduje się w \cite{DjangoDocs}. System wdrożony został przy wykorzystaniu Django 1.4.1.
    \item[Baza danych] Uruchomić i zainstalować serwer PostgreSQL (\cite{PostgreSQL}). Zainstalować rozszerzenie GIS zgodnie z instrukcją zamieszczoną w \cite{DjangoPostGIS}. System wdrożony został przy wykorzystaniu PostgreSQL 9.1.
    \item[Serwer RTMFP] Uruchomić serwer Cumulus. Instrukcja instalacji znajduje się w \cite{CumulusInstall}. Dodatkowo należy umieścić serwer na liście dostępnych serwerów w panelu administracyjnym aplikacji Django. Domyślnie, dodawane streamy losują jeden z dostępnych serwerów RTMFP. Można pokusić się o implementację algorytmu rozkładającego obciążenie bardziej równomiernie np. Round Robin.
    \item[Browser Receiver] Należy tak skonfigurować dołączanie obiektu flash, aby przekazywać informację na temat AMFGateway. Można do tego wykorzystać zmienną amfgateway dodając ją w widokach wyświetlających Browser Receiver. Domyślnie ustawiona jest na wartość \texttt{http://facewithme.com/amf/}.
    \item[Mobile Broadcaster] Należy uruchomić, zainstalować i skonfigurować środowisko Flash Develop zgodnie z tą instrukcją \cite{flashDevelopConfig}. W plikach źródłowych odnaleźć domyślną konfigurację AMFGateway (\texttt{./src/Flex/FaceWithMe Mobile/src/Main.mxml}) i poprawić ją na adres skonfigurowanego wcześniej Interfejsu WWW. Skompilować projekt oraz stworzyć odpowiednie paczki instalacyjne dla platform docelowych (Android / iOS).
\end{description}

W razie problemów z wyświetlaniem się streamu należy zapoznać się z plikiem pomocy \url{http://facewithme.com/help/}.

\section{Testy automatyczne}

Część Interfejsu WWW systemu posiada zalążki testów jednostkowych. Dotyczą one poprawnego wyświetlania widoków. Aby uruchomić testy, należy przejść do katalogu z z kodem źródłowym Django (\texttt{./src/Dajango/facewithme/}), a następnie uruchomić polecenie:

\lstset{language=Bash}
\begin{lstlisting}
python manage.py test core
\end{lstlisting}

O poprawnym wykonaniu testów informuje nas konsola, której wyjście zawiera:
\begin{lstlisting}
Creating test database for alias 'default'...
......
----------------------------------------------------------------------
Ran 6 tests in 0.822s

OK
Destroying test database for alias 'default'...
\end{lstlisting}

\section{Przeprowadzone testy}
\label{sec:EtapIprzeprowadzoneTesty}
System testowany był na kilku platformach. Niżej znajduje się lista konfiguracji systemu, w której poszczególne części nie wykazywały błędnego zachowania.

\begin{description}
    \item[Interfejs WWW z Browser Receiver] Google Chrome (21.0.1180.89) + Flash Player (11.3.31.232)/Linux; FireFox 15.0.1 + Flash Player 11.2.202.238/Linux; Google Google Chrome (21.0.1180.89) + Flash Player (11.3.31.232)/Windows 7;  FireFox 15 + Flash Player 11.2.202.238/Windows 7.
    \item[Mobile Broadcaster] Adobe AIR 3.4.0.254/Android 4.04/Samsung Galaxy S I, Adobe AIR 3.4.0.254/Android 3.2/Samsung Galaxy Tab 10.1.
\end{description}

\section{Bezpieczeństwo}
\label{sec:ImplementacjaPrototypuBezpieczenstwo}
Bardzo ważnym aspektem docelowego systemu powinno być bezpieczeństwo. Niżej wyszczególniono elementy prototypu systemu, które mają bezpośredni związek z bezpieczeństwem i sugerowane poprawki, które pomogą zabezpieczyć system.

\begin{packed_item}
    \item{Dzięki zastosowaniu technologii Adobe Flash Platform oraz wykorzystaniu protokołu RTMFP, transmisja danych pomiędzy Mobile Broadcasterem oraz Browser Receiverem jest szyfrowana. Szyfrowanie odbywa się za pomocą 128-bitowego klucza. Szczegółowe informacje można uzyskać w \cite{AdobeSecurity}}.
    \item{Komunikacja pomiędzy Interfejsem WWW, a Browser Receiverem oraz Mobile Receiverem, odbywająca się za pomocą AMFGateway powinna odbywać się za pomocą protokołu HTTPS, nie HTTP jak w prototypie.}
    \item{Logowanie do panelu administracyjnego oraz logowanie użytkowników powinno wykorzystywać protokół HTTPS, nie HTTP jak w prototypie.}
    \item{Sugeruje się przejście na HTTPS w każdej komunikacji nawiązywanej pomiędzy użytkownikiem, a Interfejsem WWW, nie HTTP jak w prototypie.}
    \item{Serwer RTMFP (Cumulus), nie udostępnia żadnego sposobu na autoryzację połączeń. Najprostszym sposobem na ograniczenie wykorzystania przez nieautoryzowanych użytkowników jest sprawdzanie \texttt{swfUrl} połączającego się klienta za pomocą rozszerzenia serwera wykorzystującego zdarzenie \texttt{onConnection} (szczegóły w \cite{CumulusDocs}).}
\end{packed_item}

\newpage
%---------------------------------------------------------------------------
