\documentclass[a4paper]{article}
\usepackage{fullpage}
\usepackage[utf8]{inputenc}
\usepackage{polski}
\usepackage[polish]{babel}

\usepackage{userstory}

\begin{document}
    \begin{userstory}{Główne zadanie systemu}
        Tutaj opisać ogólnie główne zadanie systemu, bez realizacji którego nie będzie można powiedzieć, że system działa. Jak system składa się z kilku podsystemów, dla każdego systemu przygotować podobną kartę wymagań.
        \begin{questions}
            \item{
                \textbf{Kto jest użytkownikiem systemu?} Odpowiedź
            }
            \item{
                \textbf{Jakie urządzenia muszą współpracować z systemem?} Odpowiedź
            }
            \item{
                \textbf{Czy jest wymóg użycia konkretnej technologii?} Odpowiedź
            }
            \item{
                \textbf{Jaka jest wymagana skalowalność systemu?} Odpowiedź
            }
            \item{
                \textbf{Czy są jakieś wymagania dotyczące środowiska produkcyjnego w jakim ma działać system?} Odpowiedź
            }
            \item{
                \textbf{Czy system ma współpracować z innymi systemami lub udostępniać coś innym systemom?} Odpowiedź
            }
            \item{
                \textbf{Czy istnieją systemy podobne do tworzonego?} Odpowiedź
            }
            \item{
                \textbf{Czy są jakieś inne specjalne wymagania, które nie wynikają z funkcji jakie powinien posiadać system (wymagania niefunkcjonalne)?} Odpowiedź
            }
        \end{questions}
    \end{userstory}
\end{document}
