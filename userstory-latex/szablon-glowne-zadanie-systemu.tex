\documentclass[a4paper]{article}
\usepackage{fullpage}
\usepackage[utf8]{inputenc}
\usepackage{polski}
\usepackage[polish]{babel}

\usepackage{userstory}

\begin{document}
    \begin{userstory}{Główne zadanie systemu}
        Tutaj opisać ogólnie główne zadanie systemu, bez realizacji którego nie będzie można powiedzieć, że system działa. Jak system składa się z kilku podsystemów, dla każdego systemu przygotować podobną kartę wymagań.
        \begin{questions}
            \item{
                Kto jest użytkownikiem systemu?
            }
            \item{
                Jakie urządzenia muszą współpracować z systemem?
            }
            \item{
                Czy jest wymóg użycia konkretnej technologii?
            }
            \item{
                Jaka jest wymagana skalowalność systemu?
            }
            \item{
                Czy są jakieś wymagania dotyczące środowiska produkcyjnego w jakim ma działać system?
            }
            \item{
                Czy system ma współpracować z innymi systemami lub udostępniać coś innym systemom?
            }
            \item{
                Czy istnieją systemy podobne do tworzonego?
            }
            \item{
                Czy są jakieś inne specjalne wymagania, które nie wynikają z funkcji jakie powinien posiadać system (wymagania niefunkcjonalne)?
            }
        \end{questions}
    \end{userstory}
\end{document}
