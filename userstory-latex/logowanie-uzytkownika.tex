% domyślne formatowania dokumentu, można zmienić
\documentclass[a4paper]{article}
\usepackage{fullpage}
\usepackage[utf8]{inputenc}
\usepackage{polski}
\usepackage[polish]{babel}

% dodanie wymaganego nagłówka środowiska userstory
\usepackage{userstory}

\begin{document}
    \begin{userstory}{Logowanie użytkownika}
        Użytkownik za pomocą formularza może zalogować się do serwisu,
        uzyskując w ten sposób dostęp do jego dodatkowych funkcjonalności.
        \scr{img/us1/1.png}{Formularz logowania.}
        
        \begin{tests}
            \item{
                Użytkownik tylko po podaniu podaniu loginu oraz pasującego do niego hasła
                zostaje zalogowany.
            }
            \item{
                Użytkownik po podaniu błędnego hasła lub nieistniejącego loginu,
                zostaje o tym poinformowany oraz oferuje mu się od razu formularz przypomnienia hasła.
                \scr{img/us1/2.jpg}{Ekran informacji o błędym haśle lub loginie.}
            }
            \item{
                Użytkownik po zalogowaniu
                widzi zmodyfikowane menu na wszystkich stronach serwisu.
                \scr{img/us1/2.jpg}{Zakładki menu dla zalogowanego użytkownika.}
                \scr{img/us1/3.jpg}{Zakładki menu dla niezalogowanego użytkownika.}
            }
            \item{
                Użytkownik po zalogowaniu
                ma dostęp do panelu.
            }
            \item{
                Użytkownik po zalogowaniu\\*
                może przeglądać profile innych użytkowników.
            }
        \end{tests}
        \begin{questions}
            \item{
                Czy użytkownik po wykonaniu błędnego logowania\\*
                ma być przekierowany na nową stronę, czy pozostawać na tej samej?
            }
        \end{questions}
    \end{userstory}
\end{document}
