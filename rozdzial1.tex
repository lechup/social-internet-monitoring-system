\chapter{Wprowadzenie}
\label{cha:wprowadzenie}

W ostatnich latach można zauważyć gwałtowny rozwój rozwój infrastruktury i technologii sieciowych. Coraz więcej gospodarstw domowych dołącza się do ,,chmury informacyjnej'' jaką jest Internet. Komputery bez dostępu do sieci powoli stają się po prostu bezużyteczne. Internet dzięki Wi-Fi oraz technologiom telefonii komórkowej takich jak HSPA lub jej następcy LTE staje się medium bezprzewodowym, dostępnym wszędzie tam gdzie istnieje zasięg sieci komórkowej.

Wszechobecność i dostępność ,,wielkiej pajęczyny'' powoduje, że coraz częściej korzysta się z niej nie tylko za pomocą komputerów czy laptopów. Nowe ,,gadżety'' takie jak tablety, smartphone'y czy netbooki stają się nowym interfejsem pomiędzy siecią, a użytkownikiem. Szacuje się, że do roku 2015 większość użytkowników Internetu będzie korzystała właśnie z takich urządzeń mobilnych. Wtedy również, 40\% światowej populacji (2,7 miliarda osób) będzie już korzystała z dobrodziejstw dostępu do ,,natychmiastowej informacji''. \footnote{Dane w oparciu o  artykuł \cite{Kim11}}

Zwiększenie szybkości dostępu do Internetu umożliwiło stworzenie technologii pozwalających na transmisję strumieniową danych audio/wideo. Takie serwisy jak YouTube, Vimeo czy MetaCafe już od 2005 roku umożliwiają użytkownikom wysyłanie i udostępnianie swoich nagrań online. Niestety żaden z tych serwisów nie umożliwiał udostępniania strumienia danych wideo w czasie rzeczywistym. Dopiero w 2007 roku powstał Bambuser oraz Qik. W tym roku, dosłownie parę miesięcy temu firma Google połączyła usługę YouTube z Google+ Hangouts umożliwiając równoczesne nadawanie ,,na żywo''. Wszystkie serwisy te jak i technologie, które wdrożyły, pozwalają dzielić się strumieniem audio/video -- wykorzystując do tego jedynie telefon komórkowy lub kamerę podłączoną do komputera.

Naturalną koleją rzeczy wydaje się wykorzystanie Internetu celem stworzenia gigantycznego systemu monitoringu. Można sobie wyobrazić system do którego podłączono istniejącą infrastrukturę kamer CCTV i udostępniono ich strumień video online, tak aby każdy obywatel mógł monitorować to co się dzieje w jego okolicy. System można by było rozszerzyć o kamery podłączane przez użytkowników lub ich najnowocześniejsze ,,gadżety'' czyli tablety czy smartphone'y. Tak właśnie autor niniejszej pracy wyobraża sobie ,,Społecznościowy, internetowy system monitoringu''.

\newpage
%---------------------------------------------------------------------------

\section{Cele pracy}
\label{sec:celePracy}

Głównym celem pracy, a co za tym idzie i najbardziej czasochłonnym celem pracy jest implementacja ograniczonego prototypu systemu określanego mianem ,,Społecznościowy, internetowy system monitoringu''.

Dodatkowym celem, bo koniecznym do realizacji głównego, jest stworzenie specyfikacji oraz analizy wymagań systemu określonego jako ,,Społecznościowy, internetowy system monitoringu''. Na każdym etapie przygotowywania projektu, chciano wykorzystać jak najwięcej metodyk i technologii ,,zwinnych''. Dlatego też bazując na pracy starszych kolegów \cite{JakMich06}, jak i doświadczeniu w zwinnym specyfikowaniu systemu doktorantów \cite{Mad09}, podpierając się literaturą fachową \cite{Bec99} autor chce zaproponować własną ,,zwinną specyfikację i analizę wymagań''. W szczególności pokazać jej zalety oraz wady względem podejścia konwencjonalnego. Uwypuklić przypadki, w których metodyki zwinne nie są dobrym rozwiązaniem.

\section{Zawartość pracy}
\label{sec:zawartoscPracy}

W pierwszym rozdziale pracy opisano teorię związaną ze ,,zwinną'' metodyką tworzenia oprogramowania. Umieszczono jej wady, zalety oraz wyszczególniono kiedy nie należy stosować metodyk zwinnych. Tutaj znalazło się również miejsce na opisanie idealnego cyklu życia idealnego ,,zwinnego'' projektu oraz autorską propozycję przeprowadzania akwizycji i analizy wymagań takiego przedsięwzięcia.

Kolejny rozdział pracy poświęcony jest akwizycji i analizie wymagań tytułowego ,,Społecznościowego systemu monitoringu''. W tej części pracy znajdują się więc takie elementu jak karty wymagań, ekrany, wstępna architektura systemu czy opis technologii proponowanych do zastosowania. Przy realizacji ww. elementów wykorzystano zaproponowaną i określoną w rozdziale pierwszym metodykę.

Ostatni z rozdziałów poświęcony jest najbardziej pracochłonnemu elementowi pracy -- implementacji ograniczonego prototypu systemu. Można znaleźć tutaj informacje szczegółowe na temat tego co udało się zaimplementować, z czym były problemu, jak sprawdziły się technologie i architektura wymieniona w rozdziale wcześniejszym.

\newpage
%---------------------------------------------------------------------------
