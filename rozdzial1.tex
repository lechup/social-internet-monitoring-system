\chapter{Wprowadzenie}
\label{cha:wprowadzenie}

W ostatnich latach można zauważyć gwałtowny rozwój rozwój infrastruktury i technologii sieciowych. Coraz więcej gospodarstw domowych dołącza się do ,,chmury informacyjnej'' jaką jest Internet. Komputery bez dostępu do sieci powoli stają się po prostu bezużyteczne. Internet dzięki Wi-Fi oraz technologiom telefonii komórkowej takich jak HSPA lub jej następcy LTE staje się medium bezprzewodowym, dostępnym wszędzie tam gdzie istnieje zasięg sieci komórkowej.

Wszechobecność i dostępność ,,wielkiej pajęczyny'' powoduje, że coraz częściej korzysta się z niej nie tylko za pomocą komputerów czy laptopów. Nowe ,,gadżety'' takie jak tablety, smartphone'y czy netbook'i stają się nowym interfejsem pomiędzy siecią, a użytkownikiem. Szacuje się, że do roku 2015 większość użytkowników Internetu będzie korzystała właśnie z takich urządzeń mobilnych. Wtedy również, 40\% światowej populacji (2,7 miliarda osób) będzie już korzystała z dobrodziejstw dostępu do ,,natychmiastowej informacji''. \footnote{Dane w oparciu o  artykuł \cite{Kim11}}

Zwiększenie szybkości dostępu do Internetu umożliwiło stworzenie technologii pozwalających na transmisję strumieniową danych audio/wideo. Takie serwisy jak YouTube, Vimeo czy MetaCafe już od 2005 roku umożliwiają użytkownikom wysyłanie i udostępnianie swoich nagrań online. Niestety żaden z tych serwisów nie umożliwia udostępniania strumienia danych wideo w czasie rzeczywistym. Dopiero w 2007 roku powstał Bambuser oraz Qik. Serwisy te jak i technologie, które wdrożyły, pozwalają dzielić się strumieniem audio/video -- wykorzystując do tego jedynie telefon komórkowy lub kamerę podłączoną do komputera.

Naturalną koleją rzeczy wydaje się wykorzystanie Internetu celem stworzenia gigantycznego systemu monitoringu. Można sobie wyobrazić system do którego podłączono istniejącą infrastrukturę kamer CCTV i udostępniono ich strumień video online, tak aby każdy obywatel mógł monitorować to co się dzieje w jego okolicy. System można by było rozszerzyć o kamery podłączane przez użytkowników lub ich najnowocześniejsze ,,gadżety'' czyli tablety czy smartphone'y. Tak właśnie autor niniejszej pracy wyobraża sobie ,,Społecznościowy, internetowy system monitoringu''.

\newpage
%---------------------------------------------------------------------------

\section{Cele pracy}
\label{sec:celePracy}

Głównym celem pracy jest stworzenie specyfikacji oraz analizy wymagań systemu określonego jako ,,Społecznościowy, internetowy system monitoringu''. Na każdym etapie przygotowywania projektu, chciano wykorzystać jak najwięcej metodyk i technologii ,,zwinnych''.

Bazując na pracy starszych kolegów (\cite{JakMich06}), jak i doświadczeniu w zwinnym specyfikowaniu systemu doktorantów (\cite{Mad09}), podpierając się literaturą fachową (\cite{Bec99}) autor chce zaproponować swoją ,,zwinną specyfikację i analizę wymagań''. W szczególności pokazać jej zalety oraz wady względem podejścia konwencjonalnego. Uwypuklić przypadki, w których metodyki zwinne nie są dobrym rozwiązaniem.

Dodatkowym, najbardziej czasochłonnym celem pracy jest implementacja ograniczonego prototypu specyfikowanego systemu.

\section{Zawartość pracy}
\label{sec:zawartoscPracy}

\todo

\newpage
%---------------------------------------------------------------------------

\section{,,Zwinna'' metodyka tworzenia oprogramowania -- ,,XP''}
\label{sec:ZMTO}

\begin{center}
    ,,Wytwarzając oprogramowanie i pomagając innym w tym zakresie,\\*
    odkrywamy lepsze sposoby wykonywania tej pracy.\\*
    W wyniku tych doświadczeń przedkładamy:\newline
    \newline
    \textbf{Ludzi i interakcje} ponad procesy i narzędzia.\\*
    \textbf{Działające oprogramowanie} ponad obszerną dokumentację.\\*
    \textbf{Współpracę z klientem} ponad formalne ustalenia.\\*
    \textbf{Reagowanie na zmiany} ponad podążanie za planem.\newline
    \newline
    Doceniamy to, co wymieniono po prawej stronie,\\*
    jednak bardziej cenimy to, co po lewej.''
\end{center}
\hfill \begin{small}\textit{--- Kent Beck et al., http://agilemanifesto.org}\end{small}

Przy tworzeniu projektu wchodzącego w skład niniejszej poczyniono kilka założeń. Większość z nich dotyczy sposobu prowadzenia projektu -- dokumentacji, testów, komunikacji w grupie. W następnym punkcie znajduje się lista tych założeń.

Projekt nie spełniający ich, nie będzie mógł być prowadzony ,,zwinnie''. Listę można traktować jako swojego rodzaju zbiór dobrych praktyk metodyki rozwoju oprogramowania zwanej ,,Extreme Programming'' lub w skrócie ,,XP'' -- Programowanie Ekstremalne. Oczywiście ze względu na samodzielny charakter pracy, autor nie mógł sprawdzić wszystkich założeń w praktyce. \footnote{Głównym źródłem informacji na którym  ten rozdział jest książka \cite{Bec99}. Autor tej książki uważany jest za twórcę metodyki ,,XP''.}

\subsection{Założenia}
\label{sec:ZMTOzalozenia}

\begin{packed_item}
    \item klient jest zawsze dostępny -- Osoba która wie jak system ma działać od strony użytkownika jest zawsze dostępna dla programistów, forma komunikacji nie jest narzucona, natomiast preferowana jest twarzą w twarz. Osoba ta nie jest potrzebna tylko na początku projektu lecz \emph{przez cały okres} jego tworzenia i rozwoju.
    \item klient ma zawsze możliwość zmiany -- Nie ma rzeczy której nie da się zmienić w systemie, klient może w każdym momencie zmienić wymagania systemu. Grupa projektowa musi reagować na te zmiany.
    \item projekt jest prowadzony w krótkich iteracjach -- jedno do trzy-tygodniowe okresy czasu, po których klient otrzymuje chciane funkcjonalności, wykonania których podjęliśmy się w zadanym okresie czasu.
    \item testowanie automatyczne -- Każdy nowy element systemu musi posiadać napisane testy automatyczne (jednostkowe, funkcjonalne) -- system można w każdym momencie automatycznie przetestować. Najlepiej testy pisać przed implementacją nowego elementu.
    \item projektujemy, planujemy zawsze \emph{proste minimum} -- Nie planujemy na wyrost, określamy minimum, które będzie spełniało wymagania klienta w danej iteracji. Zawsze staramy się trzymać jak najmniejszą złożoność systemu, nawet kosztem dodatkowych zmian.
    \item ciągła integracja -- Poprawki nanoszone są cały czas na istniejący, system. Integracja następuje wręcz kilka razy dziennie. Wymagany jest reżim testowy -- integrowane jest tylko ten kod, który ma napisane testy automatyczne oraz nie blokuje wykonania żadnego z dotychczasowo napisanych testów.
    \item brak specjalizacji -- Żadna osoba z grupy projektowej nie powinna specjalizować się w jakiejś funkcji (programista, architekt, integrator, analityk). Wszyscy biorą czynny udział we wszystkich aspektach projektu.
    \item grupa prowadząca projekt nie jest zbyt liczna -- Grupa w której tworzony jest projekt lub pod-projekt nie może być zbyt liczna, ułatwia to komunikację. Każdy większy projekt da się podzielić na szereg mniejszych pod-projektów. Nie ma określonego limitu górnego, jeżeli występują problemy w komunikacji najczęściej grupa projektowa jest zbyt liczna.
    \item preferowaną formą komunikacji w grupie projektowej jest komunikacja twarzą w twarz -- Tylko przy takiej rozmowie uczestnicy projektu nie są rozpraszani przez inne rzeczy i mogą skupić się na zadanym temacie. Taki sposób komunikacji ułatwia wynajdywanie błędów i niejasności we wczesnej fazie projektu.
    \item wspólne programowanie, projektowanie -- Faworyzuje się tworzenie wszystkich elementów systemu w parach. Zapewniając w ten sposób ciągłe badanie jakości kodu, czy też poprawności architektury systemu.
    \item architektura jest zmienna -- Architektura jest czymś co się zmienia wraz z rozwojem projektu i funkcjonalności wymaganej przez klienta. Może się zmienić na każdym etapie projektu.
    \item kod i testy są dokumentacją -- Nie ma prowadzonej dodatkowej dokumentacji implementacyjnej np. dla programistów. Dobrze napisane testy, kod z komentarzami oraz ostatecznie rozmowa z innymi uczestnikami projektu jest najlepszą dokumentacją aktualnego stanu systemu.
\end{packed_item}

\subsection{Zalety}
\label{sec:ZMTOzalety}

Metodyka ,,Extreme Programming'' zmienia całkowicie podejście do sposobu tworzenia oprogramowania. Głównie przez umieszczenie klienta, w centrum prowadzonego projektu i jego ciągłe zaangażowanie na całym etapie tworzenia i rozwoju systemu. To on może w dowolnym momencie wprowadzić praktycznie dowolne zmiany. Zyski jakie daje metodyka zwinna ,,XP'' można rozpatrywać na wielu płaszczyznach:

\begin{packed_item}
    \item koszty -- Metodyka daje nowe możliwości związane z liczeniem kosztów (per iteracja, per funkcjonalność). Przez krótkie iteracje i działające oprogramowanie łatwiej rozliczać się z klientem, a sam klient chętniej płaci. Długoterminowo unikamy kosztów wynikających z błędów początkowego projektowania, czy też zmian w projekcie, których nie jesteśmy w stanie uniknąć.
    \item czas -- Zarządzanie czasem w krótkich terminach iteracyjnych jest dużo bardziej wydajne i bardziej przewidywalne niż planowanie całego projektu od początku do końca. Brak konkretnej daty zakończenia projektu umożliwia uniknięcie ,,marszów śmierci'', czy przesuwania terminów. Klient sam zarządza tym co ma być zrobione i jakim kosztem czasowym jest to obarczone. Reżim testowy i ciągła integracja powoduje, że oprogramowanie dostarczane jest szybko, i co bardzo ważne dla klienta -- jest to oprogramowanie spełniające jego aktualne wymagania.
    \item jakość -- Ciągła integracja i obecność testów powoduje, że pomimo możliwych ciągłych zmian kod w większości przypadków działa, czyli jest wysokiej jakości. Jakość kodu gwarantowana jest również przez programowanie w parach. Zastępuje ono cykliczne sprawdzanie kodu oraz umożliwia uniknięcie problemów związanych z błędną architekturą systemu, które najczęściej pojawiają się bardzo późno.
    \item zakres -- Bardzo łatwo zmienić zakres systemu w trakcie jego tworzenia, wymaga się jedynie ustalenia w której iteracji ma być on zwiększony lub pomniejszony. Jako, że nie mamy ustalonych terminów całości projektu, zmiana zakresu nie jest dla nas 
    \item brak specjalizacji -- Ciągła komunikacja werbalna w projekcie zapobiega specjalizacji poszczególnych członków zespołu. Każdy powinien wiedzieć jak najwięcej o systemie. Takie podejście gwarantuje niezależność w przypadku odejścia członka ,,specjalisty''.
    \item zmniejszanie możliwości porażki projektu -- Szybkie dostarczanie działającego, funkcjonalnego oprogramowania ma duży wpływ na zadowolenie klienta. Przekłada się to bezpośrednio na poczucie satysfakcji oraz pozytywną atmosferę wśród członków zespołu. Dzięki temu ludzie są mniej skłonni do zmiany miejsca zatrudnienia ze względu na stres, ciągłe przekraczanie terminów czy brak satysfakcji z wykonywanej pracy.
\end{packed_item}

\subsection{Wady}
\label{sec:ZMTOwady}

Jak każda metodyka, także i ta ,,zwinna'' posiada swoje wady. Warto znać je przed rozpoczęciem projektu, w którym planuje się ją wykorzystać. Główne z nich to:

\begin{packed_item}
    \item ciągła dostępność klienta -- Klient lub osoba wyznaczona przez niego, która zna potrzeby systemu musi być dostępna programistom ,,na zawołanie'' i ściśle z nimi współpracować.
    \item konsekwencja -- Decydując się na ,,XP'' nie możemy zrezygnować z założeń (wspomniane szczegółowo \namedref{sec:ZMTOzalozenia}). Brak konsekwencji spowoduje w większości przypadków porażkę projektu.
    \item brak konkretnej daty zakończenia projektu -- Podejście iteracyjno-wymaganiowe powoduje, że znany jest jedynie czas zakończenia aktualnej iteracji w skład której wchodzą konkretne karty wymagań. Ciągła możliwość zmiany zakresu i funkcjonalności systemu powoduje, że nie można określić daty zakończenia projektu.
\end{packed_item}

\subsection{Kiedy nie stosować metodyki}
\label{sec:ZMTOknsm}

Czasami opisywanej metodyki nie można zastosować ze względu na charakter prowadzonego projektu lub przyzwyczajenia klienta. Główne wskazówki kiedy ,,XP'' \textbf{nie nadaje się} jako metodyka prowadzenia projektu:

\begin{packed_item}
    \item brak dostępności klienta -- Jeżeli wiadomo, że klient nie będzie brał czynnego udziału w projekcie. Doprowadzi to do sytuacji, w której tworzone oprogramowanie będzie oprogramowaniem niezgodnym z wymaganiami klienta.
    \item duża grupa projektowa -- Grupa projektowa ,,XP'' nie powinna być większa niż 10 osób. Więcej osób generuje problemy w komunikacji. Częściowym rozwiązaniem może być podział systemu na podsystemy, przygotowywane przez kilka mniejszych grup projektowych.
    \item długie wykonywanie testów -- Kompilacja i wykonanie testów automatycznych trwa dłużej niż dzień (24 godziny). Co praktycznie uniemożliwia systematyczną ,,ciągłą integrację''.
    \item brak środowiska testowego innego niż produkcyjny -- W sytuacji kiedy koszty środowiska testowego są bardzo wysokie, klient nie będzie chciał wydawać drugi raz dużych pieniędzy. Może się też zdarzyć, że z innych względów nie będziemy w stanie zagwarantować środowiska testowego, co automatycznie wyklucza testy automatyczne i ,,ciągłą integrację''.
\end{packed_item}

\subsection{Cykl życia idealnego projektu}
\label{sec:ZMTOcykl}

Każda metodyka ma etapy w jakim może znaleźć się projekt. W tym rozdziale przybliżono idealny cykl życia projektu ,,XP''.

\begin{packed_enum}
    \item \textbf{Badanie (Exploration)} Faza w której pobiera się od klienta główne wymagania systemu, testuje możliwe do wykorzystania technologie, bada się ich wydajność i użyteczność w projekcie. Jeżeli wykorzystywana technologia jest nowa dla zespołu, testuje się jej wykorzystanie na jakimś małym projekcie powiązanym z docelowym systemem.
    \item \textbf{Planowanie (Planning)} Faza w której grupa projektowa szacuje z klientem terminy głównych wymagań systemu. Klient określa priorytet, a grupa projektowa czas potrzebny na wykonanie poszczególnych kart wymagań.
    \item \textbf{Iteracje do pierwszego wydania (Iterations to First Release)} Faza w której w iteracjach jedno do trzy-tygodniowych przygotowuje się system zgodnie z kartami wymagań. Koniec każdej iteracji powinien być ,,małym świętem'' -- darmowa pizza, wolny dzień w pracy etc.
    \item \textbf{Wdrażanie do produkcji (Productionizing)} Faza w której system jest wdrażany do produkcji.Na tym etapie należy pamiętać o sposobie na łatwą integrację zmian -- zarówno kodu jak i migracji danych oraz systemie testowym. Pod koniec tego etapu należy zrobić huczną imprezę!
    \item \textbf{Utrzymanie i konserwacja (Maintenance)} Jest to najdłuższa faza projektu ,,XP''. Można by właściwie powiedzieć jego naturalny stan. W tej fazie iteracyjne, wdrażane są szczegółowe wymagania klienta, czy też nowe funkcjonalności odpowiadające jego aktualnym potrzebom.
    \item \textbf{Zakończenie (Death)} Jest to faza w której projekt uznawany jest za zakończony. Może to być sytuacja idealna, w której klient nie ma potrzeby dalszej zmiany oprogramowania -- dostarczono mu produkt idealny. Może to być również mniej przyjemna sytuacja -- system nie jest w stanie spełnić nowych wymagań stawianych przez klienta. Tak czy inaczej, dobrym pomysłem jest zorganizowanie ,,stypy'' na której trzeba poruszyć problem, tego: ,,Co można było zrobić lepiej?'' i wyciągnąć na tej podstawie wnioski mogące się przydać w nowym projekcie.
\end{packed_enum}

\subsection{,,Zwinna'' specyfikacja wymagań}
\label{cha:ZMTOzwinnaSpecyfikacjaWymagan}

%Jak wspomniano \namedref{sec:celePracy} autor chce wykorzystać jak najwięcej metodyk ,,zwinnych'' przy każdym aspekcie tworzenia oprogramowania czyli także w  przypadku specyfikacji i analizy wymagań.
%
Do tej pory nie znaleziono idealnego, uniwersalnego rozwiązania na pobieranie wymagań od klienta. W większości przypadków trzeba pewnego czasu, aby wypracować wspólny język z klientem i nauczyć się razem z nim współpracować przy akwizycji wymagań.

Podchodząc ,,zwinnie'' do projektowania systemu na początku należy pobrać od klienta tylko te wymagania, które są konieczne do osiągnięcia minimalnych celów biznesowych -- głównego zadania systemu. Wszystkie sprawy poboczne, należy oczywiście zapisywać, lecz zostawić na później, aż projekt przejdzie do etapu ,,Utrzymanie i konserwacja'' (więcej \namedref{sec:ZMTOcykl}).

Celem ,,zwinnej'' specyfikacji wymagań jest stworzenie wymagań \emph{minimalnego systemu}, spełniającego \emph{wszystkie podstawowe funkcje biznesowe} zdefiniowane przez klienta, który można \emph{wdrożyć, używać i rozwijać}.

%Na podstawowe dobre praktyki niekoniecznie ,,zwinnej'' inżynierii wymagań składają się między innymi\footnote{Informacje zaczerpnięte z \cite{Wol09p}}:

%\begin{packed_item}
%    \item Ocena wykonalności systemu -- przed przystąpieniem do szczegółowej akwizycji wymagań, należy sprawdzić wykonalność implementacji systemu
%    \item Dokument specyfikacji wymagań -- istnieje jakiś dokument, który ułatwia akwizycję wymagań
%    \begin{packed_item}
%        \item Należy zdefiniować standardową strukturę dokumentu
%        \item Należy wyjaśnić jak korzystać z dokumentu
%        \item Należy załączyć podsumowanie wymagań
%        \item Należy zadbać o czytelną strukturę dokumentu
%        \item Należy zadbać o łatwość wprowadzania zmian w dokumencie
%        \item Należy używać prostego i ścisłego języka
%        \item Należy budować słownik wymienianych terminów specjalistycznych
%    \end{packed_item}
%    \item Pozyskiwanie wymagań -- praktyki na samym etapie pozyskiwania wymagań
%    \begin{packed_item}
%        \item Należy być wyczulonym na czynniki organizacyjne i polityczne
%        \item Należy przechowywać źródła wymagań
%        \item Należy używać celów biznesowych przy pozyskiwaniu wymagań 
%    \end{packed_item}
%\end{packed_item}

\subsubsection{Narzędzia}
\label{sec:ZSWnarzedzia}

Głównymi narzędziami wykorzystywanymi przy ,,zwinnej'' specyfikacji wymagań są:
\begin{packed_item}
    \item user stories (w \cite{Mad09} określane jako karty wymagań) -- krótkie, ogólnikowe historyjki opisujące jakąś funkcjonalność systemu
    \item screens (w \cite{Mad09} określane jako ekrany) -- szkice obrazujące ogólny układ przycisków, formularzy, tabel mogących pojawić się w karcie wymagań
    \item acceptance tests (w \cite{Mad09} określane jako testy akceptacyjne/scenariusze) -- opisy słowne prawidłowego zachowania systemu dotyczące kart wymagań, swojego rodzaju scenariusz karty wymagań na bazie których programiści mogą stworzyć testy (o testach akceptacyjnych w ,,XP'' można więcej przeczytać w \cite{Jef00} pod hasłem ,,Acceptance tests'')  
\end{packed_item}

Dokumenty te nie mają konkretnej formy. ,,Zwinna'' metodyka pozwala dostosować ją do własnych wymagań, które będą działały najlepiej dla wykorzystujących metodykę programistów oraz ich klienta. Karty wymagań, scenariusze oraz ekrany należy spisywać na małych karteczkach jednego formatu. Przygotowując powyższe ,,dokumenty'' należy pamiętać o zasadach:
\begin{packed_item}
    \item prostota -- Złożone scenariusze czy karty wymagań należy rozbić na kilka mniejszych, dzielenie musi być zrobione z klientem w formie rozmowy -- nigdy przez samego programistę. Karty wymagań powinny być do wykonania w ciągu jednego do dwóch tygodni czasu programisty.
    \item klient jest autorem kart wymagań jak i testów akceptacyjnych/scenariuszy -- Programista może jedynie pokazać jak należy pisać kartę wymagań, tak żeby wypracować z klientem wspólny język, może pomagać w podzieleniu złożonych wymagań na mniejsze. Dopuszcza się sugerowanie możliwych scenariuszy, natomiast sama reakcja systemu musi być już określona przez klienta.
    \item ekrany powinny mieć swoją sekwencję -- Jak istnieje sekwencja ekranów, należy pamiętać o nadaniu numerów odpowiadających kolejności ich występowania.
\end{packed_item}

\subsubsection{Cyfryzacja i zarządzanie}
\label{sec:ZSWcyfryzacja}

W związku z potrzebą dołączenia kart wymagań, związanych z nimi testów akceptacyjnych oraz ekranów do niniejszej pracy, trzeba było wymyślić sposób na ich cyfryzację. Dodatkową zaletą formy elektronicznej jest fakt, że ułatwia ona wymianę informacji pomiędzy uczestnikami projektu, pomoże uniknąć sytuacji zagubienia lub zniszczenia jakiegoś elementu kart wymagań, ułatwi ich zarządzaniem i przetrzymywaniem np. w wykorzystywanym systemie kontroli wersji.

Wykorzystywanie \LaTeX~u do składu pracy nakierowało na pomysł stworzenia szablonu \LaTeX, który można by było wykorzystać celem konwersji kart wymagań i ich elementów analogowych do dokumentu cyfrowego. Łatwe oddzielenie warstwy prezentacji od samej treści oraz możliwość eksportu do formatu PDF umożliwiałoby również tworzenie w prosty sposób ładnego dokumentu dla klienta po spotkaniach na których pobierane są od niego wymagania.

Tak powstał tzw. ,,package'' udostępniający środowisko \texttt{userstory}. Szczegółowy opis tego środowiska znajduje się \namedref{cha:dodatekA}.

\subsection{,,Zwinna'' analiza wymagań}
\label{sec:ZMTOzwinnaAnalizaWymagan}

Zwinna analiza wymagań ma miejsce w fazie ,,Badanie (Exploration)'' projektu (więcej \namedref{sec:ZMTOcykl}), po tym jak ustalono z klientem główne zadanie systemu oraz sporządzono dla niego wszystkie karty wymagań. Celem tej fazy jest:

\begin{packed_item}
    \item testowanie technologii -- Poprzez budowanie prototypów, testuje się wykonalność oraz wydajność konkretnej technologii. Testy służą wybraniu najbardziej odpowiedniej technologii. Jeżeli uruchomienie jakiejś technologii trwa dłużej niż tydzień, należy zastanowić się nad alternatywami.
    \item testowanie architektury -- Najczęściej system można zbudować na kilka sposobów, z pomocą może przyjść znowu prototypownie. W grupach 2-3 osobowych tworzone są równolegle 2-3 podejścia do architektury systemu. Porównanie umożliwia wybranie tej najlepszej.
    \item poznanie limitów wydajnościowych rozwiązania -- Bardzo ważnym czynnikiem, który trzeba wziąć pod uwagę jest wydajność i odnieść ją wymagań stawianych przez klienta.
    \item testowanie skalowalności -- Równie ważnym czynnikiem jest sposób w jaki skaluje się projektowane rozwiązanie. Zapewnienie stałej, liniowej, bądź logarytmicznej skalowalności jest ideałem.
    \item wybór narzędzi umożliwiających ,,ciągłą integrację'' -- Jednym z ważniejszych wymagań ,,XP'' jest zdolność do ciągłej integracji. Należy wybrać system kontroli wersji kodu źródłowego, narzędzie do przeprowadzania testów automatycznych oraz narzędzie umożliwiające łatwą i pewną migrację danych.
    \item wybór narzędzi wspomagających prowadzenie projektu -- Powinno się również pomyśleć o narzędziach wspomagających zarządzanie projektem, czy też ułatwiającymi komunikację między uczestnikami projektu.
\end{packed_item}

Zgodnie z założeniami ,,XP'' (więcej \namedref{sec:ZMTOzalozenia}), wszystkie ustalenia sporządzone w tej fazie mogą ulec zmianie na każdym późniejszym etapie prowadzenia projektu. Zwinna analiza wymagań nie zamyka możliwości późniejszej zmiany technologii, architektury czy dowolnego z elementów wspomagających projekt.

\newpage
%---------------------------------------------------------------------------
