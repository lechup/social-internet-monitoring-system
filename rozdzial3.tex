\chapter{Implementacja prototypu systemu}
\label{cha:ImplementacjaPrototypu}

\section{Co udało się zaimplementować}

Prototyp systemu dostępny jest on-line pod adresem \url{http://facewithme.com}. Przy tworzeniu prototypu systemu skupiono się na 4 elementach systemu (szczegóły architektury znajdują się \namedref{sec:EtapIwstepnaArchitekturaSystemu}):
\begin{packed_item}
    \item{Interfejs WWW}
    \item{baza danych}
    \item{serwer RTMFP}
    \item{Mobile Broadcaster}
    \item{Browser Receiver}
\end{packed_item}

Dzięki wybraniu ww. elementów udało się stworzyć system zdolny do przetestowania transmisji wideo pomiędzy Mobile Broadcaster, a Browser Receiver, za pomocą technologii Adobe P2P Multicast. Interfejs WWW, umożliwia dodatkowo ładną prezentację oraz wyszukiwanie streamów w zadanej lokalizacji czy też personalizację ustawień. Niżej przedstawiono listę elementów systemu wraz z funkcjami jakie udalo się zaimplementować.

\large{\textbf{Interfejs WWW}}
\begin{packed_item}
    \item{system rejestracji użytkownika (\url{http://facewithme.com/accounts/register})}



    \item{system logowania użytkownika (\url{http://facewithme.com/accounts/login})}
    \item{panel administracyjny do zarządzania stream'ami, kategoriami i użytkownikami (\url{http://facewithme.com/admin})}
    \item{interaktywną mapę prezentującą stream'y (\url{http://facewithme.com})}
    \item{funkcję automatycznej lokalizacji użytkownika i ustawienie pozycji mapy zgodnie z nią (\url{http://facewithme.com})}
    \item{listowanie wszystkich streamów nadawanych w systemie z podziałem na kategorie  (\url{http://facewithme.com/stream/list})}
\end{packed_item}

\section{Co trzeba usprawnić}

Przy implementacji prototypu systemu, nie skupiano się na sprawach pobocznych jak interfejs czy komunikacja z użytkownikiem. Głównie kierowano się intencją przetestowania w praktyce działania technologii Adobe P2P Multicast. Dlatego też system wymaga znacznej ilości poprawek zanim może zącząć być wykorzystywany publicznie. Przygotowany prototyp traktować można jako bazę wyjściową do zapoznania się z działaniem i wydajnością technologii. Główne braki systemu to:

\begin{packed_item}
    \item{brak szyfrowania połączeń HTTP pomiędzy użytkownikiem a systemem, dotyczy także logowania}
    \item{brak jakiejkolwiek autoryzacji serwera RTMFP, jest dostępny publicznie}
    \item{brak zabezpieczenia publicznego/prywatnego streamu wideo}
    \item{brak całego Mobile Receiver'a}
    \item{brak całego Browser Broadcaster'a}
    \item{panel DEBUG w Mobile Broadcasterz'e}
    \item{brak obrotu komponentu kamery względem obrócenia urządzenia w Mobile Broadcasterz'e}
    \item{brak automatycznego usuwania stream'u z Interfejsu WWW w przypadku nieoczekiwanego przerwania działania Mobile Broadcaster'a}
    \item{brak komunikatów skierowanych do użytkownika Mobile Broadcaster'a, informujących o stanie lub błędach aplikacji}
    \item{brak wyłączania trybu czuwania}
\end{packed_item}


\section{Dokumentacja}
Zgodnie z filozofią ,,zwinnej metodyki'' (szczegóły \namedref{sec:ZMTOzalozenia}), dokumentacja projektu sprowadza się do opisu architektury oraz komentarzy kodu, testów oraz komentarzy do testów. Unika się tworzenia bazy wiedzy, która najczęściej jest po pewnym czasie nieaktualizowana.

Aby uruchomić system z kodu źródłowego należy skonfigurować 

\section{Testowanie}

Tutaj jak i na jakich urządzeniach / przeglądarkach było testowane...

\section{Kod źródłowy}

\section{Bezpieczeństwo}

\section{Wydajność P2P w Adobe Flash Platform}

\section{Pomysły na rozwój}

\section{Wnioski}


\newpage
%---------------------------------------------------------------------------
